\documentclass[10pt,a4paper,onecolumn]{article}
\usepackage{marginnote}
\usepackage{graphicx}
\usepackage{xcolor}
\usepackage{authblk,etoolbox}
\usepackage{titlesec}
\usepackage{calc}
\usepackage{tikz}
\usepackage{hyperref}
\hypersetup{colorlinks,breaklinks,
            urlcolor=[rgb]{0.0, 0.5, 1.0},
            linkcolor=[rgb]{0.0, 0.5, 1.0}}
\usepackage{caption}
\usepackage{tcolorbox}
\usepackage{amssymb,amsmath}
\usepackage{ifxetex,ifluatex}
\usepackage{seqsplit}
\usepackage{fixltx2e} % provides \textsubscript
\usepackage[
  backend=biber,
%  style=alphabetic,
%  citestyle=numeric
]{biblatex}
\bibliography{paper.bib}



% --- Page layout -------------------------------------------------------------
\usepackage[top=3.5cm, bottom=3cm, right=1.5cm, left=1.0cm,
            headheight=2.2cm, reversemp, includemp, marginparwidth=4.5cm]{geometry}

% --- Default font ------------------------------------------------------------
% \renewcommand\familydefault{\sfdefault}

% --- Style -------------------------------------------------------------------
\renewcommand{\bibfont}{\small \sffamily}
\renewcommand{\captionfont}{\small\sffamily}
\renewcommand{\captionlabelfont}{\bfseries}

% --- Section/SubSection/SubSubSection ----------------------------------------
\titleformat{\section}
  {\normalfont\sffamily\Large\bfseries}
  {}{0pt}{}
\titleformat{\subsection}
  {\normalfont\sffamily\large\bfseries}
  {}{0pt}{}
\titleformat{\subsubsection}
  {\normalfont\sffamily\bfseries}
  {}{0pt}{}
\titleformat*{\paragraph}
  {\sffamily\normalsize}


% --- Header / Footer ---------------------------------------------------------
\usepackage{fancyhdr}
\pagestyle{fancy}
\fancyhf{}
%\renewcommand{\headrulewidth}{0.50pt}
\renewcommand{\headrulewidth}{0pt}
\fancyhead[L]{\hspace{-0.75cm}\includegraphics[width=5.5cm]{/home/grayson/R/x86_64-pc-linux-gnu-library/4.3/rticles/rmarkdown/templates/joss/resources/JOSS-logo.png}}
\fancyhead[C]{}
\fancyhead[R]{}
\renewcommand{\footrulewidth}{0.25pt}

\fancyfoot[L]{\footnotesize{\sffamily White, (2024). gglm: An R package
implementing the grammar of graphics for linear model diagnostic
plots. \textit{Journal of Open Source Software}, (), . \href{https://doi.org/}{https://doi.org/}}}


\fancyfoot[R]{\sffamily \thepage}
\makeatletter
\let\ps@plain\ps@fancy
\fancyheadoffset[L]{4.5cm}
\fancyfootoffset[L]{4.5cm}

% --- Macros ---------

\definecolor{linky}{rgb}{0.0, 0.5, 1.0}

\newtcolorbox{repobox}
   {colback=red, colframe=red!75!black,
     boxrule=0.5pt, arc=2pt, left=6pt, right=6pt, top=3pt, bottom=3pt}

\newcommand{\ExternalLink}{%
   \tikz[x=1.2ex, y=1.2ex, baseline=-0.05ex]{%
       \begin{scope}[x=1ex, y=1ex]
           \clip (-0.1,-0.1)
               --++ (-0, 1.2)
               --++ (0.6, 0)
               --++ (0, -0.6)
               --++ (0.6, 0)
               --++ (0, -1);
           \path[draw,
               line width = 0.5,
               rounded corners=0.5]
               (0,0) rectangle (1,1);
       \end{scope}
       \path[draw, line width = 0.5] (0.5, 0.5)
           -- (1, 1);
       \path[draw, line width = 0.5] (0.6, 1)
           -- (1, 1) -- (1, 0.6);
       }
   }

% --- Title / Authors ---------------------------------------------------------
% patch \maketitle so that it doesn't center
\patchcmd{\@maketitle}{center}{flushleft}{}{}
\patchcmd{\@maketitle}{center}{flushleft}{}{}
% patch \maketitle so that the font size for the title is normal
\patchcmd{\@maketitle}{\LARGE}{\LARGE\sffamily}{}{}
% patch the patch by authblk so that the author block is flush left
\def\maketitle{{%
  \renewenvironment{tabular}[2][]
    {\begin{flushleft}}
    {\end{flushleft}}
  \AB@maketitle}}
\makeatletter
\renewcommand\AB@affilsepx{ \protect\Affilfont}
%\renewcommand\AB@affilnote[1]{{\bfseries #1}\hspace{2pt}}
\renewcommand\AB@affilnote[1]{{\bfseries #1}\hspace{3pt}}
\makeatother
\renewcommand\Authfont{\sffamily\bfseries}
\renewcommand\Affilfont{\sffamily\small\mdseries}
\setlength{\affilsep}{1em}


\ifnum 0\ifxetex 1\fi\ifluatex 1\fi=0 % if pdftex
  \usepackage[T1]{fontenc}
  \usepackage[utf8]{inputenc}

\else % if luatex or xelatex
  \ifxetex
    \usepackage{mathspec}
  \else
    \usepackage{fontspec}
  \fi
  \defaultfontfeatures{Ligatures=TeX,Scale=MatchLowercase}

\fi
% use upquote if available, for straight quotes in verbatim environments
\IfFileExists{upquote.sty}{\usepackage{upquote}}{}
% use microtype if available
\IfFileExists{microtype.sty}{%
\usepackage{microtype}
\UseMicrotypeSet[protrusion]{basicmath} % disable protrusion for tt fonts
}{}

\usepackage{hyperref}
\hypersetup{unicode=true,
            pdftitle={gglm: An R package implementing the grammar of graphics for linear model diagnostic plots},
            pdfborder={0 0 0},
            breaklinks=true}
\urlstyle{same}  % don't use monospace font for urls
\usepackage{graphicx,grffile}
\makeatletter
\def\maxwidth{\ifdim\Gin@nat@width>\linewidth\linewidth\else\Gin@nat@width\fi}
\def\maxheight{\ifdim\Gin@nat@height>\textheight\textheight\else\Gin@nat@height\fi}
\makeatother
% Scale images if necessary, so that they will not overflow the page
% margins by default, and it is still possible to overwrite the defaults
% using explicit options in \includegraphics[width, height, ...]{}
\setkeys{Gin}{width=\maxwidth,height=\maxheight,keepaspectratio}
\IfFileExists{parskip.sty}{%
\usepackage{parskip}
}{% else
\setlength{\parindent}{0pt}
\setlength{\parskip}{6pt plus 2pt minus 1pt}
}
\setlength{\emergencystretch}{3em}  % prevent overfull lines
\setcounter{secnumdepth}{0}
% Redefines (sub)paragraphs to behave more like sections
\ifx\paragraph\undefined\else
\let\oldparagraph\paragraph
\renewcommand{\paragraph}[1]{\oldparagraph{#1}\mbox{}}
\fi
\ifx\subparagraph\undefined\else
\let\oldsubparagraph\subparagraph
\renewcommand{\subparagraph}[1]{\oldsubparagraph{#1}\mbox{}}
\fi


% tightlist command for lists without linebreak
\providecommand{\tightlist}{%
  \setlength{\itemsep}{0pt}\setlength{\parskip}{0pt}}


% Pandoc citation processing
\newlength{\cslhangindent}
\setlength{\cslhangindent}{1.5em}
\newlength{\csllabelwidth}
\setlength{\csllabelwidth}{3em}
\newlength{\cslentryspacingunit} % times entry-spacing
\setlength{\cslentryspacingunit}{\parskip}
% for Pandoc 2.8 to 2.10.1
\newenvironment{cslreferences}%
  {}%
  {\par}
% For Pandoc 2.11+
\newenvironment{CSLReferences}[2] % #1 hanging-ident, #2 entry spacing
 {% don't indent paragraphs
  \setlength{\parindent}{0pt}
  % turn on hanging indent if param 1 is 1
  \ifodd #1
  \let\oldpar\par
  \def\par{\hangindent=\cslhangindent\oldpar}
  \fi
  % set entry spacing
  \setlength{\parskip}{#2\cslentryspacingunit}
 }%
 {}
\usepackage{calc}
\newcommand{\CSLBlock}[1]{#1\hfill\break}
\newcommand{\CSLLeftMargin}[1]{\parbox[t]{\csllabelwidth}{#1}}
\newcommand{\CSLRightInline}[1]{\parbox[t]{\linewidth - \csllabelwidth}{#1}\break}
\newcommand{\CSLIndent}[1]{\hspace{\cslhangindent}#1}


\newenvironment{cols}[1][]{}{}

\newenvironment{col}[1]{\begin{minipage}{#1}\ignorespaces}{%
\end{minipage}
\ifhmode\unskip\fi
\aftergroup\useignorespacesandallpars}

\def\useignorespacesandallpars#1\ignorespaces\fi{%
#1\fi\ignorespacesandallpars}

\makeatletter
\def\ignorespacesandallpars{%
  \@ifnextchar\par
    {\expandafter\ignorespacesandallpars\@gobble}%
    {}%
}
\makeatother

\title{gglm: An R package implementing the grammar of graphics for
linear model diagnostic plots}

        \author[1]{Grayson W. White}
    
      \affil[1]{Michigan State University, Department of Forestry}
  \date{\vspace{-5ex}}

\begin{document}
\maketitle

\marginpar{
  %\hrule
  \sffamily\small

  {\bfseries DOI:} \href{https://doi.org/}{\color{linky}{}}

  \vspace{2mm}

  {\bfseries Software}
  \begin{itemize}
    \setlength\itemsep{0em}
    \item \href{}{\color{linky}{Review}} \ExternalLink
    \item \href{}{\color{linky}{Repository}} \ExternalLink
    \item \href{}{\color{linky}{Archive}} \ExternalLink
  \end{itemize}

  \vspace{2mm}

  {\bfseries Submitted:} \\
  {\bfseries Published:} 

  \vspace{2mm}
  {\bfseries License}\\
  Authors of papers retain copyright and release the work under a Creative Commons Attribution 4.0 International License (\href{http://creativecommons.org/licenses/by/4.0/}{\color{linky}{CC-BY}}).
}

\hypertarget{summary}{%
\section{Summary}\label{summary}}

\texttt{gglm} implements an interface to produce publication-ready model
diagnostic plots that complies with the grammar of graphics (Wickham,
2010). Further, \texttt{gglm} utilizes the \texttt{broom} and
\texttt{broom.mixed} R packages to provide support for diagnostic plots
produced from a variety of model object classes across a wide variety of
R packages (Bolker \& Robinson, 2022; Robinson, Hayes, \& Couch, 2023).
A quartet of diagnostic plots can be quickly created using
\texttt{gglm}'s homonymous function, or through instructive and
intuitive layer functions added to a \texttt{ggplot2} object (Wickham,
2016).

\hypertarget{statement-of-need}{%
\section{Statement of need}\label{statement-of-need}}

When scientists, statistical practitioners, students, and others
implement statistical models, it is of the utmost importance that the
modeling assumptions are verified through visual diagnostics in order to
ensure valid statistical inference. The R statistical software language
provides a method for producing diagnostic plots for linear model
objects created with \texttt{stats::lm}, however these plots are
visually unappealing, inconsistent with diagnostic plots produced for
other R packages and model types, and out of place in modern statistics
and data science courses focused on learning R with the
\texttt{tidyverse} (Wickham et al., 2019).

\texttt{gglm} addresses the described issues with current diagnostic
plots in R by providing a consistent interface for producing beautiful
and publication-ready diagnostic plots for a large variety of R packages
and model types (linear models, linear mixed models, generalized linear
mixed models, etc.). \texttt{gglm} provides functionality to quickly
produce four common diagnostic plots, similar to
\texttt{stats::plot.lm}, but produced by \texttt{ggplot2}. Further,
\texttt{gglm} provides a suite of layer functions adhering to the
grammar of graphics which allow the user to create and fine-tune their
diagnostic plots through \texttt{ggplot2}'s intuitive interface. The
layer functions are particularly applicable in modern courses teaching
linear regression where students have already learned \texttt{ggplot2},
and in particular they are used in Harvard University's introductory
statistics course (McConville, 2023). Outside of educational benefits,
\texttt{gglm} has potential to allow researchers to more easily publish
elegant diagnostic plots. \texttt{gglm} has been downloaded from CRAN
over 23,000 times as of January 2024.

\hypertarget{usage-and-philosophy}{%
\section{Usage and Philosophy}\label{usage-and-philosophy}}

\texttt{gglm} has a simple philosophy for usage of the package: ``be
easy, intuitive, and customizable''. This philosophy comes about from
the understanding that an individual producing a diagnostic plot will be
in one of two camps: 1) the individual who wants an \emph{easy} to use
tool that allows them to quickly check their model diagnostics, or 2)
the individual who wants an \emph{intuitive and customizable} tool that
allows them to look closely at their diagnostics for the purposes of
education, fine-tuning for publication, or other reasons. \texttt{gglm}
satisfies the individuals in both camps.

The \texttt{gglm::gglm} function is made for folks in the first camp who
are looking a more aesthetically pleasing alternative to
\texttt{stats::plot.lm}. In practice, the process of using
\texttt{gglm::gglm} is as simple as and more general than using
\texttt{stats::plot.lm}, with steps as follows:

\begin{itemize}
\tightlist
\item
  fit a model of any class listed in
  \texttt{gglm::list\_model\_classes},
\item
  call \texttt{gglm::gglm} on the saved model object.
\end{itemize}

The \texttt{gglm::stat\_*} functions are thus for the individual in the
second camp. \texttt{gglm} provides seven functions of this sort,
including those that produce the following plots: Cook's distance by
leverage, Cook's distance by observation number, fitted values by
residual values, normal QQ, residual histogram, residual values by
leverage, and scale by location. The steps to produce a diagnostic plot
with these functions are more fluid than with \texttt{gglm::gglm}, but
are easy to understand provided the user has an understanding of how to
use \texttt{ggplot2}. One may use the workflow:

\begin{itemize}
\tightlist
\item
  fit a model of any class listed in
  \texttt{gglm::list\_model\_classes},
\item
  provide the saved model object as data to \texttt{ggplot2::ggplot},
\item
  add their intended diagnostic plot layer,
\item
  add any more \texttt{ggplot2} layers such as themes, labels,
  annotations, and more to create their custom diagnostic plot.
\end{itemize}

\hypertarget{comparison-to-other-packages}{%
\section{Comparison to Other
Packages}\label{comparison-to-other-packages}}

Functionality similar to that of \texttt{gglm}'s is provided by a
variety of R packages. As mentioned throughout, \texttt{stats} provides
a \texttt{plot} method for producing diagnostic plots for \texttt{lm}
objects with base R graphics (R Core Team, 2023). Further,
\texttt{lindia} produces diagnostic plots for \texttt{lm} objects with
\texttt{ggplot2} graphics, but does not include functions that adhere
with the grammar of graphics. Finally, many packages provide methods for
plotting diagnostics based on their own model classes (see,
e.g.~\texttt{lme4::plot.merMod}), however these methods are do not have
consistent usage across packages (Bates, Mächler, Bolker, \& Walker,
2015).

\hypertarget{references}{%
\section*{References}\label{references}}
\addcontentsline{toc}{section}{References}

\hypertarget{refs}{}
\begin{CSLReferences}{1}{0}
\leavevmode\vadjust pre{\hypertarget{ref-lme4}{}}%
Bates, D., Mächler, M., Bolker, B., \& Walker, S. (2015). Fitting linear
mixed-effects models using {lme4}. \emph{Journal of Statistical
Software}, \emph{67}(1), 1--48.
doi:\href{https://doi.org/10.18637/jss.v067.i01}{10.18637/jss.v067.i01}

\leavevmode\vadjust pre{\hypertarget{ref-broom.mixed}{}}%
Bolker, B., \& Robinson, D. (2022). \emph{Broom.mixed: Tidying methods
for mixed models}. Retrieved from
\url{https://CRAN.R-project.org/package=broom.mixed}

\leavevmode\vadjust pre{\hypertarget{ref-mcconville2023}{}}%
McConville, K. (2023). STAT 100: Introduction to statistics and data
science. Harvard University Department of Statistics. Retrieved from
\url{https://mcconvil.github.io/stat100f23/}

\leavevmode\vadjust pre{\hypertarget{ref-R}{}}%
R Core Team. (2023). \emph{R: A language and environment for statistical
computing}. Vienna, Austria: R Foundation for Statistical Computing.
Retrieved from \url{https://www.R-project.org/}

\leavevmode\vadjust pre{\hypertarget{ref-broom}{}}%
Robinson, D., Hayes, A., \& Couch, S. (2023). \emph{Broom: Convert
statistical objects into tidy tibbles}. Retrieved from
\url{https://CRAN.R-project.org/package=broom}

\leavevmode\vadjust pre{\hypertarget{ref-wickham2010}{}}%
Wickham, H. (2010). A layered grammar of graphics. \emph{Journal of
Computational and Graphical Statistics}, \emph{19}(1), 3--28.
doi:\href{https://doi.org/10.1198/jcgs.2009.07098}{10.1198/jcgs.2009.07098}

\leavevmode\vadjust pre{\hypertarget{ref-ggplot2}{}}%
Wickham, H. (2016). \emph{ggplot2: Elegant graphics for data analysis}.
Springer-Verlag New York. Retrieved from
\url{https://ggplot2.tidyverse.org}

\leavevmode\vadjust pre{\hypertarget{ref-tidyverse}{}}%
Wickham, H., Averick, M., Bryan, J., Chang, W., McGowan, L. D.,
François, R., Grolemund, G., et al. (2019). Welcome to the {tidyverse}.
\emph{Journal of Open Source Software}, \emph{4}(43), 1686.
doi:\href{https://doi.org/10.21105/joss.01686}{10.21105/joss.01686}

\end{CSLReferences}

\end{document}
